\chapter{Implementation and Validation}

\indent\indent From Chapter 2 onwards, every chapter should start with an introduction paragraph. This paragraph should brief about the flow of the chapter. This introduction can be limited within 4 to 5 sentences. The chapter heading should be appropriately modified (a sample heading is shown for this chapter).But don't start the introduction paragraph in the chapters 2 to end with "This chapter deals with....". Instead you should bring in the highlights of the chapter in the introduction paragraph. 

\section{Contents of this chapter}
This chapter should elaborate the following in detail.
\begin{enumerate}
\item Implementation details for hardware based projects
\item Top level Design for software based projects
\end{enumerate}
\section{Power Delivery Controller}
You can add sections and sub sections to elaborate your project work done.
\subsection{RTL Design}
\subsection{Synthesis}
\subsection{Place and Route}
\subsection{Timing Analysis}

\vspace{0.75cm}

 \textbf{The chapters should not end with figures, instead bring the paragraph explaining about the figure at the end followed by a summary paragraph.}
\section{Power Supply}
After elaborating the various sections of the chapter (From Chapter 2 onwards), a summary paragraph should be written discussing the highlights of that particular chapter. This summary paragraph should not be numbered separately. This paragraph should connect the present chapter to the next chapter. 
\subsection{Voltage Regulation}
\subsection{Current Regulation}
\section{Register Bank}
\section{Integration}
\subsection{System Integration}
\subsection{Interface Design}
\subsection{Integration Challenges}
%\section{Validation and Testing}
